\begin{titlepage}
\thispagestyle{empty}
\pagenumbering{gobble}
\begin{center}

  \vspace{.5cm}
  \huge{Comparison of link layer of BLE and 802.15.4}\tiny{}\\
  \vspace{0.3cm}
  \begin{tikzpicture}
	\draw[gray,ultra thick] (0,0) -- (14,0);
  \end{tikzpicture}\\
  \vspace{0.2cm}
  \LARGE{\textit{Running on Contiki OS}}\\
  \vspace{2cm}
  \Large{Master Thesis}\\
  

  \vspace{2cm}		
  \LARGE{PrithviRaj Narendra}
  \vspace{3.5cm} 
  
  \large{\textit{Supervisor}}\\
  \LARGE{Simon Duquennoy}\\
  \vspace{0.2cm}
  \Large{Senior Researcher, SICS}\\
  \vspace{2cm}
  
  \large{\textit{Academic Examiner}}\\
  \LARGE{Mats Brorsson}\\
  \vspace{0.2cm}
  \Large{Professor, KTH}\\
  \vspace{2cm}

\large
EIT ICT Labs Master School Embedded Systems Program\\
School of Information and Communication Technology\\
KTH Royal Institute of Technology\\
Stockholm, Sweden\\
\vspace{1cm}
\displaydate{date}
\\
\end{center} 

\end{titlepage}

\newpage\null\thispagestyle{empty}\newpage

\thispagestyle{plain}
\pagenumbering{roman}
\phantomsection
\addcontentsline{toc}{chapter}{Abstract}
\huge{\textbf{Abstract}} \\
\normalsize \\


There has been extensive research in the low power Wireless Sensor Network (WSN) community with 802.15.4 based platforms. A major factor for this is the support for 802.15.4 based platforms in lightweight Operating Systems (OS) for Internet of Things (IoT) devices. Bluetooth Low Energy (BLE) with its standardized protocol and wide adoption in mobile devices is well suited to all applications requiring direct interaction with a mobile device. BLE does not have support in any of these software platforms for IoT development. With this as motivation this thesis creates a port of Contiki OS to a BLE platform, specifically a platform based on nrf51822 System on Chip (SoC). This will enable direct communication of Contiki nodes with smart-phones and ease development of BLE based projects with Contiki.

This thesis extends the research on BLE by comparing its link layer with 802.15.4's ContikiMAC and Null-RDC on four metrics, namely data rate, latency, reliability and energy consumption. Their behavior with and without external WiFi interference also has been looked into. The tests conducted showcases the performance of simple point to point communication 802.15.4, which is rarely benchmarked in research community that prefers testing complex topologies. The effect of the limits of the number of packets communicated per connection interval in different BLE stacks can be seen on the data rate achievable with BLE. The influence of frequency hopping on the reliability of BLE communication with the presence of external interference is assessed. Adaptive Frequency Hopping (AFH) has been emulated by manually choosing interference free channel map and its effect on mitigating interference has been evaluated. Tests also assesses the impact of BLE link layer configuration, especially creating an asymmetric connection by using non zero slave latency value on latency and energy consumption. With this asymmetric connection, the slave devices have been recorded to a latency of 16 ms with Radio Duty Cycle (RDC) of 0.6\%. %In the same test suite the 802.15.4 queried node had a latency of 24 ms (100\% RDC) and 90 ms (1.3\% RDC) when using Null-RDC and ContikiMAC respectively.



%Internet of Things (IoT) is at the peak of 2014's `Hype Cycle for Emerging Technologies'. Fervent work is being carried out to achieve this vision of billions of interconnected smart devices by researchers, companies and hobbyists. This thesis project focuses on two key technologies fueling the development of IoT devices, namely Bluetooth Low Energy (BLE) and Contiki Operating System (OS). More specifically, this thesis aims to port Contiki OS to a BLE based hardware platform and compare the link layer of BLE with 802.15.4 based Medium Access Control (MAC) layers used in Contiki. 

%The process of porting began with compiling a list of `must have' and `nice to have' requirements for the BLE platform to which Contiki would be ported. An exhaustive list of features of the available platforms was aggregated and based on the requirements identified, the nrf51822 System on Chip (SoC) based platform was chosen. The vendor of this platform provided a BLE stack as a binary, which was used in this project. All the basic features of a Contiki port such as the Contiki clock for etimer, rtimer, serial-line and LEDs have implementations for the chosen platform. \todo{The radio peripheral was left out since it was controlled by the stack in the binary, although an additional task of developing an BLE advertisement packet logger has been done from scratch.} This port has been utilized for a demo application and for the test cases described below.

%The primary research output of this thesis is the comparison of performance of the link layer of BLE with 802.15.4 based ContikiMAC and Null-RDC layers of Contiki in terms of data rate, reliability, latency and energy consumption. Two test suites were designed to collect data of these four metrics and compare the two protocols. Some test cases also included presence of external interference caused by presence of heavy WiFi traffic. The scenario assumed is where one device is unconstrained with respect to availability of power while the other is not.

%In terms of the data rate, the 802.15.4's NullRDC layer achieved the highest at 155 kbps without Clear Channel Assesment (CCA). The effect of WiFi interference overlapping in the channel of communication in case 802.15.4 can be seen when the data rate reduced to 61 kbps from 148 kbps in the case where interference was in a different channel. With 802.15.4 the data rate can be maximized by having Radio Duty Cycle (RDC) of 100\%. The influence of frequency hopping in BLE can be noticed when the data rate decreased to only 23 kbps from 29 kbps when WiFi interference was introduced, both cases having RDC of around 27\%.  In BLE the effect of packets communicated per connection interval can be seen when communicating with an Android device as the data rate increased to 86 kbps, in which RDC was 44\%. 

%Use of 802.15.4's CCA delivered a Packet Reception Ratio (PRR) of 99\% and 79\% with and without WiFi interference respectively. In case of BLE, the link layer's the simple acknowledgment scheme achieved a Packet Delivery Ratio (PDR) of 100\%. Without WiFi interference, BLE achieved a PRR of almost 100\%. When WiFi interference was introduced, PRR resurged back to greater than 99\% when WiFi free channel map was used as compared to 80\% when complete channel map, which shows working of Adaptive Frequency Hopping (AFH).

%802.15.4's latency was measured as 24 ms with Null-RDC and 90 ms with ContikiMAC. With BLE latency values ranging from 14 ms to 750 ms were observed. These tests to measure latency show the influence of link layer parameters such as `connection interval' and `slave latency', as well as whether the data is queried by the master node or the slave node. The change in symmetry of the connection with non-zero slave latency can be seen in case where the latency changes from 16 ms to 736 ms when the test case changes from the slave querying the data to the master querying the data. The master's RDC of 22\% and slave's RDC of 0.6\% also indicates the asymmetry in the connection. With these different configurations, the different use cases are also illustrated.

\clearpage

%\newpage\null\thispagestyle{empty}\newpage
%
%\thispagestyle{plain}
%\phantomsection
%\addcontentsline{toc}{chapter}{Sammanfattning}
%\huge{\textbf{Sammanfattning}} \\
%\normalsize \\
%Hello Sammanfattning.
%\clearpage

\newpage\null\thispagestyle{empty}\newpage

\thispagestyle{plain}
\phantomsection
\addcontentsline{toc}{chapter}{Acknowledgment}
\huge{\textbf{Acknowledgment}} \\
\normalsize \\


Foremost I would like to earnestly thank my supervisor at Swedish Institute of Computer Science (SICS), Simon Duquennoy, for agreeing to supervise me in a topic that I proposed, guiding me patiently, consistently reviewing and providing constructive feedback for my work. Next I would like to thank Thiemo Voigt and the Networked Embedded Systems group at SICS for their warm inclusiveness and assisting me whenever I needed help. As usual this open source work stands on the shoulders of giants, I'm thankful for their amazing work. For all the resources and guidance that I have received in various forms from my fellow netizens, I'm grateful to them. I thank EIT ICT Labs for believing in me and providing me this opportunity. As always, my family and friends have supported and encouraged me, and for that they have my heartfelt thanks.
\clearpage

\newpage\null\thispagestyle{empty}\newpage

\tableofcontents
\null
\vfill
\begin{wrapfigure}{r}{0.25\textwidth}
\vspace{-30pt}
  \begin{center}
	\includegraphics[width=0.22\textwidth]{cc-by}
  \end{center}
\end{wrapfigure}
\noindent
{\large This work is licensed under a \href{http://creativecommons.org/licenses/by/4.0/}{Creative Commons Attribution 4.0 International License}.}


\glsnogroupskiptrue
\renewcommand{\glsnamefont}[1]{\textbf{#1}}
\setlength{\glsdescwidth}{0.8\hsize}
\printglossary[style=long,title=List of Abbreviations,type=\acronymtype]
\clearpage

\addcontentsline{toc}{chapter}{\listfigurename}
\listoffigures
\clearpage
\addcontentsline{toc}{chapter}{\listtablename}
\listoftables
\clearpage