\chapter[Background]{Background\footnote{Includes text and images from minor thesis 'Business ideation for \gls{ble}'}}

\section{Overview of \acrlong{ble}}
Rest of overview, from what is missed in 1st chapter.

\subsection{Design objectives of \gls{ble}}
\gls{ble} was designed by Bluetooth \gls{sig} from the ground up, which helped it achieve certain design goals. These design goals for this wireless personal area network were \emph{low cost, worldwide operation, short range, robustness} and \emph{low power}\cite{Heydon2012}. 
\paragraph{Low Power} \gls{ble} aims to use a tiny batteries such as button cell to keep a device operating for months to years. To achieve this goal, \gls{ble} was optimized to communicate small amounts of data, such as the states of devices. Also \gls{ble} is optimized to have lower peak power requirements, which allows use of button cells to be used with \gls{ble} devices.
\paragraph{Worldwide Operation}
For a technology to be adopted, it is important that there is uniform conformity to the regulations around the world. The 2.4 \si{\GHz} \gls{ism} radio band is the only one available license free worldwide. The technology to develop wireless devices in this band is mature making it the suitable radio band for \gls{ble}.
\paragraph{Short Range}
\gls{ble} was designed to be for personal area network like Classic Bluetooth, which means that it is not a network to work with a cellular base station network. This design criterion goes hand in hand with \emph{low power}.
\paragraph{Low Cost} Lower power requirements mean that the batteries in \gls{ble} devices need to smaller and have to be replaced less frequently, both resulting in a reduction of cost for both the manufacturer and the customer. The use of the \gls{ism} band for communication levels removes the licensing entry barrier for start-ups to develop \gls{ble} devices. \gls{ble} embraces simplicity in its pursuit to lower the cost. \gls{ble} supports only single-hop communication in a star network, which reduces the memory and processor requirement for supporting the protocol. Simplicity was the key factor for the choosing of \gls{gfsk} as the modulation scheme for \gls{ble} to result in low cost, small radio implementation of the radio in the \glspl{ic} for \gls{ble}. 
\paragraph{Robustness} The 2.4~\si{\GHz} space is crowded with devices communicating with various standards as well as spurious noise making the robustness a key criteria in developing \gls{ble} standard. \gls{ble} uses a multi-channel hopping mechanism called \gls{afh} to detect, avoid and recover from interference. In addition to \gls{afh}, \gls{ble} uses \gls{crc} to detect and recover from bit-errors due to background noise.

\subsection{\gls{ble} network architecture}
Single and Dual devices,
Advertisers and scanners,
Master and Slave,
Star Network,
figure \ref{devicesBLE}
\begin{figure}[h] %{r}{0.51\textwidth}
%\vspace{-15pt}
  \begin{center}
	\includegraphics[width=0.49\textwidth]{devicesBLE}
  \end{center}
\caption{Typical BLE network}
%\vspace{-10pt}
\label{devicesBLE}
\end{figure}

\section{\gls{ble} stack overview}
Explain host and controller division. More details about \gls{ll}.

\textbf{Image of the stack}

\paragraph{Physical layer}
\paragraph{\acrfull{ll}}
\paragraph{\gls{l2cap}}
\paragraph{\gls{gap}}
\paragraph{\gls{att}}
\paragraph{\gls{gatt}}
\paragraph{\gls{sm}}

\section{Overview of Contiki}
??????????????? Which aspects of Contiki here?

\section{Overview of 802.15.4}
Mention that in this thesis the Contiki specific implementations of the 802.15.4 layers will be tested.

\subsection{Physical layer}

\subsection{\gls{mac} layer}
\subsubsection{\gls{rdc} layer}
\subsubsection{\gls{csma}}

\todo{Seperate section Previous work done on comparing \gls{ble} and 802.15.4}


