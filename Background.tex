\chapter[Background]{Background\footnote{Contains text \& images from the Minor Thesis of the author `Business Ideation for \gls{ble}', derived from this thesis' work}} \label{2Back}

Research articles involving the process of porting of Contiki \gls{os} to a hardware platform are presented in section \ref{2Porting}.


\section{Porting the Contiki \gls{os} to a New Platform} \label{2Porting}
Contiki \gls{os} project originating in 2002 has been ported to an increasing number of hardware platforms \cite{contikiHw}. Since Contiki is an open-source BSD Clause-3 licensed project, there are many projects in various hardware platforms based on a fork from the Contiki repository.

These hardware platforms' processors range across a spectrum of 8-bit (8051 and AVR), 16-bit (MSP430) to 32-bit (ARM Cortex-M, PIC32). Contiki has support for 802.15.4 based wireless communication with various external transceivers and \glspl{soc} with built in radio transceiver. Various common features of many platforms such as \glspl{led}, buttons and serial port have modules in Contiki for common \gls{api} across platforms.

In \cite{Oikonomou2011} the authors describe Contiki's port to a CC2430 based platform manufactured by Sensinode Ltd. CC2430 is an enhanced Intel 8051 processor based \gls{soc} having 802.15.4 physical layer compatible radio transceiver. The authors fully debugged the port which had of a code footprint of about 100 kB, varying based on the compiler mode and the features enabled. Many new features were added to the port including support for \gls{adc} unit, all the sensor available on the platform (accelerometer, light sensor, voltage and temperature sensors), watchdog timer and the general purpose buttons. Because of the limited stack availability of 233 bytes, many optimizations such as moving variables to external \gls{ram} memory space and re-writing the radio driver for CC2430 to prevent stack from overflowing.

Contiki was ported to two new platforms, namely MicaZ and TelosB in a Bachelor thesis \cite{stan2007porting}. TelosB, similar to the TMote-Sky platform, consists of a 8MHz 16-bit MSP430 \gls{mcu}, a CC2400 transceiver with 802.15.4 \gls{phy} and a host of sensors to measure light, temperature and humidity. The port to TelosB was done by changing the port to the fully supported TMote-Sky platform. MicaZ platform consisted of a 8-bit Atmel ATMega128L \gls{mcu} and a CC2400 transceiver. MicaZ platform needed rewriting of the code for processor abstraction so that the high level Contiki \glspl{api} could work.

Contiki has been ported into platform based on a ARM Cortex M3 processor to create a device wired with Ethernet to connected to the Internet \cite{Wilde2013a}. Ethernet based networking with the libraries present in Contiki was developed in this project to demonstrate as a proof of concept.

There are many unofficial ports of Contiki to various hardware platforms, including ones based on ARM Cortex M3 based processors. This can be seen in the port to STM32F10x based platform \cite{Padrah} and LPC1768 based platform \cite{Tanyingyong2013}, both having Cortex M3 processor.



\section{Overview of \acrlong{ble}}
Rest of overview, from what is missed in 1st chapter. control and monitoring applications especially in the healthcare, fitness security and home entertainment industry.

\subsection{Design objectives of \gls{ble}}
\gls{ble} was designed by Bluetooth \gls{sig} from the ground up, which helped it achieve certain design goals. These design goals for this wireless personal area network were \emph{low cost, worldwide operation, short range, robustness} and \emph{low power}\cite{Heydon2012}. 
\paragraph{Low Power} \gls{ble} aims to use a tiny batteries such as button cell to keep a device operating for months to years. To achieve this goal, \gls{ble} was optimized to communicate small amounts of data, such as the states of devices. Also \gls{ble} is optimized to have lower peak power requirements, which allows use of button cells to be used with \gls{ble} devices.
\paragraph{Worldwide Operation}
For a technology to be adopted, it is important that there is uniform conformity to the regulations around the world. The 2.4 \si{\GHz} \gls{ism} radio band is the only one available license free worldwide. The technology to develop wireless devices in this band is mature making it the suitable radio band for \gls{ble}.
\paragraph{Short Range}
\gls{ble} was designed to be for personal area network like Classic Bluetooth, which means that it is not a network to work with a cellular base station network. This design criterion goes hand in hand with \emph{low power}.
\paragraph{Low Cost} Lower power requirements mean that the batteries in \gls{ble} devices need to smaller and have to be replaced less frequently, both resulting in a reduction of cost for both the manufacturer and the customer. The use of the \gls{ism} band for communication levels removes the licensing entry barrier for start-ups to develop \gls{ble} devices. \gls{ble} embraces simplicity in its pursuit to lower the cost. \gls{ble} supports only single-hop communication in a star network, which reduces the memory and processor requirement for supporting the protocol. Simplicity was the key factor for the choosing of \gls{gfsk} as the modulation scheme for \gls{ble} to result in low cost, small radio implementation of the radio in the \glspl{ic} for \gls{ble}. 
\paragraph{Robustness} The 2.4~\si{\GHz} space is crowded with devices communicating with various standards as well as spurious noise making the robustness a key criteria in developing \gls{ble} standard. \gls{ble} uses a multi-channel hopping mechanism called \gls{afh} to detect, avoid and recover from interference. In addition to \gls{afh}, \gls{ble} uses \gls{crc} to detect and recover from bit-errors due to background noise.

\subsection{\gls{ble} network architecture}
Single and Dual devices,
Advertisers and scanners,
Master and Slave,
Star Network,
figure \ref{devicesBLE}
\begin{figure}[h] %{r}{0.51\textwidth}
%\vspace{-15pt}
  \begin{center}
	\includegraphics[width=0.49\textwidth]{devicesBLE}
  \end{center}
\caption{Typical BLE network}
%\vspace{-10pt}
\label{devicesBLE}
\end{figure}

\section{\gls{ble} stack overview}
Explain host and controller division. More details about link layer.

\textbf{Image of the stack}

The link layer of 802.15.4 consists of \gls{mac} layer on top of a \gls{rdc} layer. For the RDC layer the Null-RDC and ContikiMAC driver will be tested in each of the test with \gls{csma} as the \gls{mac} layer. In case of Contiki-MAC the receiving node switches on periodically to sense if there are any packets that need to be received. The default time of this period is 125 ms. In case of null-RDC the radio receiver is never switched off, as the name suggests.

In \gls{ble}, the devices can assume different roles in the different layers of the protocol. In the link layer a device can be a 'Master' or a 'Slave'. In the \gls{att} layer, a device can be a 'Client' and/or 'Server'. A server contains data and the client can request data from the server.

\paragraph{Physical layer}
\paragraph{Link layer}
\paragraph{\gls{l2cap}}
\paragraph{\gls{gap}}
\paragraph{\gls{att}}
\paragraph{\gls{gatt}}
\paragraph{Security Manager}

\section{Overview of Contiki}
??????????????? Which aspects of Contiki here?

\section{Overview of 802.15.4}
Mention that in this thesis the Contiki specific implementations of the 802.15.4 layers will be tested.

\subsection{Physical layer}

\subsection{\gls{mac} layer}
\subsubsection{\gls{rdc} layer}
\subsubsection{\gls{csma}}


