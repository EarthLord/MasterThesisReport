\chapter{Literature Study}

\section{Porting Contiki \gls{os} to a new platform}
Contiki \gls{os} project originating in 2002 has been ported to an increasing number of hardware platforms\cite{contikiHw}. Since Contiki is an open-source BSD Clause-3 licensed project, there are many projects in various hardware platforms based on a fork from the Contiki repository.

These hardware platforms' processors range across a spectrum of 8-bit (8051 and AVR), 16-bit (MSP430) to 32-bit (ARM Cortex-M, PIC32). Contiki has support for 802.15.4 based wireless communication with various external transceivers and \glspl{soc} with built in radio transceiver. Various common features of many platforms such as \glspl{led}, buttons and serial port have modules in Contiki for common \gls{api} across platforms.

In \cite{Oikonomou2011} the authors describe Contiki's port to a CC2430 based platform manufactured by Sensinode Ltd. CC2430 is an enhanced Intel 8051 processor based \gls{soc} having 802.15.4 physical layer compatible radio transceiver. The authors fully debugged the port which had of a code footprint of about 100 \gls{kb}, varying based on the compiler mode and the features enabled. Many new features were added to the port including support for \gls{adc} unit, all the sensor available on the platform (accelerometer, light sensor, voltage and temperature sensors), watchdog timer and the general purpose buttons. Because of the limited stack availability of 233 bytes, many optimizations such as moving variables to external \gls{ram} memory space and re-writing the radio driver for CC2430 to prevent stack from overflowing.

Contiki was ported to two new platforms, namely MicaZ and TelosB in a Bachelor thesis \cite{stan2007porting}. TelosB, similar to the TMote-Sky platform, consists of a 8MHz 16-bit MSP430 \gls{mcu}, a CC2400 transceiver with 802.15.4 \gls{phy} and a host of sensors to measure light, temperature and humidity. The port to TelosB was done by changing the port to the fully supported TMote-Sky platform. MicaZ platform consisted of a 8-bit Atmel ATMega128L \gls{mcu} and a CC2400 transceiver. MicaZ platform needed rewriting of the code for processor abstraction so that the high level Contiki \glspl{api} could work.

Contiki has been ported into platform based on a ARM Cortex M3 processor to create a device wired with Ethernet to connected to the Internet \cite{Wilde2013a}. Ethernet based networking with the libraries present in Contiki was developed in this project to demonstrate as a proof of concept.

There are many unofficial ports of Contiki to various hardware platforms, including ones based on ARM Cortex M3 based processors. This can be seen in the port to STM32F10x based platform \cite{Padrah} and LPC1768 based platform \cite{Tanyingyong2013}, both having Cortex M3 processor.

\section{Evaluating the performance of \gls{ble}}
In a paper \cite{Gomez2012} providing an overview and evaluation of \gls{ble}, its various parameters such as energy consumption, latency, maximum piconet size and throughput have been evaluated. Theoretical calculations were done to predict the lifetime of a \gls{ble} slave device running of a 230 mAh coin cell battery for different values of \emph{connection interval} and \emph{slave latency}. The average current consumption of a TI CC2540 \gls{ble} platform was plotted for the entire range of connection interval keeping the slave latency as zero. The estimate of the lifetime was also done with respect to various \gls{ber}.

\todo{Talk about latency wrt Gomez2012}
The same experimental setup \cite{Gomez2012} shows the maximum throughput between two CC2540 devices as 58.48 kbps. This is attributed to the fact that in each connection event with an interval of 7.5 ms four packets are not transmitted as in an ideal case. An analytical model of the throughput \cite{Gomez2011} has been developed for different \gls{ber} and connection interval. Simulation results validate this model developed.

The energy consumption for an advertiser and scanner is modeled for the activity of the scanner discovering the advertiser \cite{liu2012energy}. The current consumption for each phase of advertisement and scanning is measured. Using this information the a model of the energy consumption is developed taking into account the advertising and scanning interval. The model developed is compared with experiments and validated. 





Energy \cite{Kindt2014} \cite{liu2012energy}

Throughput, Energy
\todo{decode whats said in Mackensen2012}
\cite{Mackensen2012}


\section{Evaluating the performance of 802.15.4}

\section{Comparison of BLE and 802.15.4}

Low power\cite{Siekkinen2012} SimpliciTI\cite{Mikhaylov2013} ZigbeeANT \cite{Dementyev2013}
